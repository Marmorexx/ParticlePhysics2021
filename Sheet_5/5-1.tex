\documentclass{article}

\usepackage[margin=4cm]{geometry}
\usepackage{fancyhdr}
    \pagestyle{fancy}
\usepackage{fontspec}
    \setsansfont{Linux Biolinum O}
\usepackage{polyglossia}
    \setmainlanguage{english}
\usepackage{sectsty}
    \sectionfont{\normalfont\sffamily\bfseries\color{blue!40!black}}
    \subsectionfont{\normalfont\sffamily\bfseries\color{blue!30!black}}
\usepackage{amsmath}
\usepackage{amssymb}
\usepackage{siunitx}
\usepackage{float}
\usepackage{booktabs}
\usepackage{subcaption}
\usepackage{graphicx}
\usepackage{xcolor}
\usepackage{listings}
    \lstset{language=Python,
	basicstyle=\footnotesize\ttfamily,
	breaklines=true,
	framextopmargin=50pt,
	frame=bottomline,
	backgroundcolor=\color{white!86!black},
	commentstyle=\color{blue},
	keywordstyle=\color{red},
	stringstyle=\color{orange!80!black}}
\usepackage{tikz}
\usepackage{hyperref}

\title{\textsf{\color{blue!40!black}Particle Physics - Exercise Sheet 05}}
\author{Maurice Donner \and Jan Hubrich}

\begin{document}
\maketitle

\section{Polarized beams}

\textbf{a)} Since we are at high energies, we can assume that \( E \gg m \).\\
The chiral nature of QED states, that only two of the four helicity combinations
lead to a non-zero 4-vector current. The initial and final state vector currents
are
\begin{align}
    \nonumber e ^{-} &: \theta = \varphi = 0 \quad 
    \nonumber e ^{+} : \theta = \varphi = \pi \\
    \nonumber j _{(e ) , \text{RL}} &= 2 E \left( 0, -1, i , 0 \right) \\
    \nonumber j _{(e) , \text{RR}} &= ( 0, 0, 0, 0) \\
    \nonumber j _{(e) , \text{LL}} &= ( 0, 0, 0, 0) \\
    \nonumber j _{(e ) , \text{LR}} &= 2 E \left( 0, -1, -i , 0 \right) \\
    \nonumber \mu ^{-} &: \theta , \varphi = 0 \quad 
    \nonumber \mu ^{+} : \theta - \pi , \varphi = \pi \\
    \nonumber j _{(\mu ) , \text{RL}} &= 2 E \left( 0, - \cos \left( \theta \right)
    \nonumber , i , \sin \left( \theta \right) \right) \\
    \nonumber j _{(\mu) , \text{RR}} &= ( 0, 0, 0, 0) \\
    \nonumber j _{(\mu) , \text{LL}} &= ( 0, 0, 0, 0) \\
    j _{(\mu) , \text{LR}} &= 2 E \left( 0, - \cos \left( \theta \right)
    , -i , \sin \left( \theta \right) \right) 
\end{align}
which lead to the non-zero matrix elements
\begin{align}
    \left| M _{\text{RL} \rightarrow \text{RL}} \right|
    &= - \frac{e^2}{q^2} \ j _{e , \text{RL}} \ g _{\mu \nu} \
    j _{\mu , \text{RL}} =
    e^4 \left[ 1 + \cos \left( \theta \right) \right] \\
    \left| M _{\text{RL} \rightarrow \text{LR}} \right|
    &= - \frac{e^2}{q^2} \ j _{e , \text{RL}} \ g _{\mu \nu} \
    j _{\mu , \text{LR}} =
    e^4 \left[ 1 - \cos \left( \theta \right) \right] \\
    \left| M _{\text{LR} \rightarrow \text{RL}} \right|
    &= - \frac{e^2}{q^2} \ j _{e , \text{LR}} \ g _{\mu \nu} \
    j _{\mu , \text{RL}} =
    e^4 \left[ 1 - \cos \left( \theta \right) \right] \\
    \left| M _{\text{LR} \rightarrow \text{LR}} \right|
    &= - \frac{e^2}{q^2} \ j _{e , \text{LR}} \ g _{\mu \nu} \
    j _{\mu , \text{LR}} =
    e^4 \left[ 1 + \cos \left( \theta \right) \right]
\end{align}
Lastly, we will use the formula for the total cross section to compare
production rates
\begin{align}
    \label{eq:CS}
    \frac{d \sigma}{d \Omega} = \frac{1}{64 \pi ^2 s} \frac{p ^{*}_f}{p ^{*}_{i}}
    \int \left| M _{fi} \right| d \Omega
\end{align}
\begin{itemize}
    \item If both beams are unpolarized, we have to consider all possible
	initial to final state configurations
	\begin{align}
	\langle \left| M _{fi} \right| \rangle &=
	\frac{1}{4} \left(
	\left| M _{\text{RL} \rightarrow \text{RL}} \right| ^2 +
	\left| M _{\text{RL} \rightarrow \text{LR}} \right| ^2 +
	\left| M _{\text{LR} \rightarrow \text{RL}} \right| ^2 +
	\left| M _{\text{LR} \rightarrow \text{LR}} \right| ^2
	\right)
    \end{align}
Plugging into (\ref{eq:CS}) yields the known cross section for the annihilation
process
\begin{align}
    \sigma = \frac{4 \pi \alpha ^{2} }{3s}
\end{align}
    \item If the electron beam has left-handed polarization, one of the two
	allowed initial state is not possible. Therefore there are only
	two possible states in total, so we only need to average over two
	states this time
	\begin{align}
	\langle \left| M _{fi} \right| \rangle &=
	\frac{1}{\color{red}{2}} \left(
	\left| M _{\text{LR} \rightarrow \text{RL}} \right| ^2 +
	\left| M _{\text{LR} \rightarrow \text{LR}} \right| ^2
	\right)
	\end{align}
	The total cross section is the same as for an unpolarized beam:
	\( \sigma = 4 \pi \alpha ^{2} / 3 s \)
    \item If the electron beam is left-handed and the positron beam is
	right handed, there is only one possible initial state. However,
	there are two possible final states:	
	\begin{align}
	\langle \left| M _{fi} \right| \rangle &=
	\frac{1}{\color{red}{1}} \left(
	\left| M _{\text{LR} \rightarrow \text{RL}} \right| ^2 +
	\left| M _{\text{LR} \rightarrow \text{LR}} \right| ^2
	\right)
	\end{align}
	Hence the total cross section yields \( 8 \pi \alpha ^2 / 3s \) and is
	a facter two larger than for the unpolarized beam.
    \item Having both the electron and the positron beam polarized into
	the same direction, yields an initial state that is forbidden
	\begin{align}
	    \left| M _{\text{LL}} \right| = 
	    \underbrace{\left| M _{\text{LL}\rightarrow \text{LL}} \right|}_{=0} + 
	    \underbrace{\left| M _{\text{LL}\rightarrow \text{LR}} \right|}_{=0} + 
	    \underbrace{\left| M _{\text{LL}\rightarrow \text{RL}} \right|}_{=0} + 
	    \underbrace{\left| M _{\text{LL}\rightarrow \text{RR}} \right|}_{=0}  
	    = 0
	\end{align}
	Therefore the cross section of the matrix element \( \left| M _{fi} 
	    \right| \) is also equal to zero. No production is observed.
\end{itemize}
\textbf{b)} To calculate the fraction of right- to left-handed \( \mu ^{-} \),
we look exemplary at the integral over the single matrix elements
\begin{align}
    \nonumber \int \left| M _{\text{LR} \rightarrow \text{LR}} \right|^2 =
    \int_{-1}^{1} e ^{4} \left( 1+ \cos \left( \theta \right) ^2\right)
    d \left( \cos \theta \right) = \frac{8}{3} \\
    \int \left| M _{\text{LR} \rightarrow \text{RL}} \right|^2 =
    \int_{-1}^{1} e ^{4} \left( 1- \cos \left( \theta \right) ^2\right)
    d \left( \cos \theta \right) = \frac{8}{3} \\
\end{align}
The fraction of \( \mu ^{-} _{\downarrow} \) and \( \mu ^{-} _{\uparrow} \)
is therefore equal to 1.

\end{document}
