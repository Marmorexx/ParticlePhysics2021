\documentclass{article}

\usepackage[margin=4cm]{geometry}
\usepackage{fancyhdr}
    \pagestyle{fancy}
\usepackage{fontspec}
    \setsansfont{Linux Biolinum O}
\usepackage{polyglossia}
    \setmainlanguage{english}
\usepackage{sectsty}
    \sectionfont{\normalfont\sffamily\bfseries\color{blue!40!black}}
    \subsectionfont{\normalfont\sffamily\bfseries\color{blue!30!black}}
\usepackage{amsmath}
\usepackage{amssymb}
\usepackage{siunitx}
\usepackage{float}
\usepackage{booktabs}
\usepackage{subcaption}
\usepackage{graphicx}
\usepackage{xcolor}
\usepackage{listings}
    \lstset{language=Python,
	basicstyle=\footnotesize\ttfamily,
	breaklines=true,
	framextopmargin=50pt,
	frame=bottomline,
	backgroundcolor=\color{white!86!black},
	commentstyle=\color{blue},
	keywordstyle=\color{red},
	stringstyle=\color{orange!80!black}}
\usepackage{tikz}
\usepackage{hyperref}

\title{\textsf{\color{blue!40!black}Particle Physics - Exercise Sheet 02}}
\author{Maurice Donner \and Jan Hubrich}

\begin{document}

\section{Cross Sections}
\textbf{a)} The differential cross-section for \( e^+e^- \rightarrow
    \mu^+\mu^-\) scattering can be described as
\begin{align}
    \frac{d \sigma}{d \Omega} = \frac{\alpha^2}{4s} ( 1 + \cos^2 \theta )
\end{align}
Using \( d \Omega = d \varphi d (\cos \theta) \), we get the total cross-section
\begin{align}
    \sigma &= \frac{2 \pi \alpha ^2}{4s} \int ( 1 + \cos^2 \theta )
    d(\cos \theta) \\
    \label{eq:integral}
    &= \frac{2 \pi \alpha^2}{4s} \left( \cos \theta \Big| _{-1} ^{1} +
	\frac{1}{3} \cos^3 \theta \Big| _{-1} ^{1} \right) \\
    \label{eq:sigma}
    &= \frac{2 \pi \alpha ^2}{4 s} \cdot \frac{8}{3} = \frac{4 \pi \alpha^2}{3s}
\end{align}
The luminosity is \( \mathcal{L} = 10 ^{30} \ \si{\centi
	\meter^{-2} \second ^{-1} } \). Expressed in natural units, that is
\begin{align}
\mathcal{L} &= 10 ^{4} \ \si{\femto \meter ^{-2} \second ^{-1}}
\cdot (\hbar c)^2 \\
&= 10 ^{4} \cdot 197^2 \ \si{\mega \electronvolt ^2 \second ^{-1}}
\end{align}
With \( \alpha = 1/137 \), \( s = (10 ^{4} \ \si{\mega \electronvolt})^2 
= 10 ^{8} \ \si{\mega \electronvolt ^2}\), and (\ref{eq:sigma}), we get the
production rate
\begin{align}
    \eta = 10 ^{4} \cdot 197 ^{2} \frac{4 \pi}{3 \cdot 137 ^{2} \cdot 10 ^{8}}
    \ \si{\second ^{-1}} = 0.87 \cdot 10 ^{-3} \ \si{\second ^{-1}} = 
    74.8 \ \si{d ^{-1} }
\end{align}
\textbf{b)} Instead of integrating from -1 to 1 in (\ref{eq:integral}), we
integrate from \( \cos(150 ^{\circ} ) = -\sqrt{3} /2\) to 
\( \cos(30 ^{\circ} ) = \sqrt{3} /2 \). That yields
\begin{align}
    \sigma = \frac{2 \pi \alpha ^2}{4s} \left( \sqrt{3} + \frac{2}{3} 
	\frac{\sqrt{3} ^{3}}{8} \right) = \frac{5 \sqrt{3} ^{3}}{12}
    \frac{2 \pi \alpha ^2}{4s} = \frac{5 \sqrt{3} ^{3} \pi \alpha ^2}{24 s}
\end{align}
Multiplying with the efficiency \( \varepsilon = 0.9 \) the actual detection
rate is
\begin{align}
    \eta' = 0.8 \cdot 10 ^{4} \cdot 197 ^{2} \frac{5 \sqrt{3} ^{3} \pi}{24
	\cdot 137^2 \cdot 10 ^{8} }
    \ \si{\second ^{-1}} = 0.63 \cdot 10 ^{3} \ \si{\second ^{-1}} = 54.7
    \ \si{d ^{-1}}
\end{align}
The actual detection efficiency is therefore \( 73 \% \).

\end{document}
