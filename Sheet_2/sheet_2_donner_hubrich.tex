\documentclass{article}

\usepackage[margin=4cm]{geometry}
\usepackage{fancyhdr}
    \pagestyle{fancy}
\usepackage{fontspec}
    \setsansfont{Linux Biolinum O}
\usepackage{polyglossia}
    \setmainlanguage{english}
\usepackage{sectsty}
    \sectionfont{\normalfont\sffamily\bfseries\color{blue!40!black}}
    \subsectionfont{\normalfont\sffamily\bfseries\color{blue!30!black}}
\usepackage{amsmath}
\usepackage{amssymb}
\usepackage{siunitx}
\usepackage{float}
\usepackage{booktabs}
\usepackage{subcaption}
\usepackage{graphicx}
\usepackage{xcolor}
\usepackage{listings}
    \lstset{language=Python,
	basicstyle=\footnotesize\ttfamily,
	breaklines=true,
	framextopmargin=50pt,
	frame=bottomline,
	backgroundcolor=\color{white!86!black},
	commentstyle=\color{blue},
	keywordstyle=\color{red},
	stringstyle=\color{orange!80!black}}
\usepackage{tikz}
\usepackage{hyperref}

\title{\textsf{\color{blue!40!black}Particle Physics - Exercise Sheet 02}}
\author{Maurice Donner \and Jan Hubrich}

\begin{document}

\maketitle
\newpage
\section{Neutrinos in matter}
The mean free path is described by
\begin{align}
    \lambda = \frac{1}{\sigma n}
\end{align}
Where \( n \) is the target density of a material. Using \( \sigma =
8 \cdot 10 ^{-39} \ \si{\centi \meter ^2}\) and \( n = \rho N_A/A \), where
\( N_A = 6.022 \cdot 10 ^{23}\) is the Avogadro constant, \( \rho = 
7.9 \cdot 101 ^{3} \ \si{\kilo \gram \per \meter ^3}\), the density of iron
and \( A = 56 \) its atomic number, we get
\begin{align}
    \lambda = \frac{1}{\sigma n} = \frac{A}{\sigma \rho N_A} = 1.471
    \cdot 10 ^{12} \ \si{\meter}
\end{align}
This means, that the neutrino experience one interaction after a distance
of \( \lambda \), which means the probability of interaction in \( l = 2m \) of
iron is \( 1.36 \cdot 10 ^{-12} \). 


\section{Phase space integration}
The interaction cross section \( \sigma \) for a scattering process
\( a+b \rightarrow 1+2 \) can be described by 
\begin{align}
    \label{eq:sig}
    \sigma = \frac{1}{(2 \pi ^2 )} \frac{1}{4 p_i^{*} \sqrt{s}} \int
    \left| M_{fi} \right|^2 \delta ( \sqrt{s} -E_1 - E_2 )
    \delta ^3 ( \vec p_1+\vec p_2 ) \frac{\vec p_1}{2 E_1} \frac{\vec p_2}{2 E_2}
\end{align}
First, we use
\begin{align}
    \delta(f(x)) &= \left| \frac{df}{dx} \right| ^{-1} _{x=x_0} \ \text{with}\ f(x_0) = 0 \\
    \text{and} \ E_1 &= \sqrt{m_1^2+p_1^2} \ \text{and}\ E_2 = \sqrt{m _{2} ^{2} +p _{2}^{2} }
\end{align}
to rewrite the first delta term and evaluate it for \( p_2 = -p_1 \):
\begin{align}
    \delta \left( \sqrt{s} - E_1 - E_2 \right) &= \nonumber
    \delta \left( \sqrt{s} - \sqrt{m _{1}^2 + p_1^2 } - \sqrt{m _{2} ^{2} + p _{2} ^2} \right) \\
    &= \left| - \frac{2 p_1}{2 \sqrt{m_1^2 +p_1^2}} - \frac{2 p_1}{2 \sqrt{m_2^2 +p_1^2}} \nonumber
    \right| ^{-1} \\
    &= \left( \frac{p_1}{E_1} + \frac{p_1}{E_2} \right) ^{-1}
    = \left(\frac{ p_1E_2+p_1E_1}{E_1E_2} \right)^{-1} \nonumber
    \\ &= \frac{E_1 E_2}{p_1 \sqrt{s}}
    \label{eq:delta}
\end{align}
Now we transform (\ref{eq:sig}) into spherical coordinates and use (\ref{eq:delta})
\begin{align}
    \text{with} \ d^3 \vec{p_i} &= p_i^2 dp_i d \Omega \nonumber \\
    \Rightarrow \sigma &= \frac{p_f^{*}}{16 \pi ^2 p_i^{*} \sqrt{s}} 
    \frac{E_1E_2}{4 E_1E_2 \sqrt{s}} \int
    \left| M_{fi} \right|^2 d \Omega^{*} \nonumber \\
    &= \frac{p_f^{*}}{64 \pi ^2 p_i^{*} s} \int
    \left| M_{fi} \right|^2 d \Omega^{*}
\end{align}

\section{Fermi's golden rule}
\textbf{b)} The energy of the product particles has to match the rest mass of
\( \eta \):
\begin{align}
    m_ \eta &= E_p + E _{\bar p} = 2 E_p \nonumber \\
    \Rightarrow \frac{m_\eta^2}{4} &= m_p^2 + p_f^2 \quad
    \Rightarrow \quad p_f = \left(\frac{m_\eta^2}{4} - m_p^2 \right) ^{\frac{1}{2}}
\end{align}
\textbf{c)} Fermi's golden rule states that
\begin{align}
\Gamma _{fi} = \frac{p_f^{*}}{32 \pi ^2 m_A*2} \int \left| M_{fi} \right| d \Omega^{*}
\end{align}
Assuming a similar matrix element, the ratio of the two decays is 
\begin{align}
    \frac{\Gamma _{\eta \rightarrow p + \bar p}}{\Gamma _{\eta \rightarrow \Lambda + \bar \Lambda}}
	= \frac{\sqrt{m _{\eta}^2-4m_p^2} }{\sqrt{m _{\eta}^2-4m_\Lambda^2} }
	= \frac{33.29 \ \si{\mega \electronvolt}}{27.46 \ \si{\mega \electronvolt}} = 1.17
\end{align}
The pdg databook suggests a ratio of 1.35, thus the matrix element of the proton
decay has to be larger than the matrix element of the \( \Lambda \) decay. \\
\textbf{d)} With \( \tau = 1/\Gamma \) we get \( \tau = 2.34 \cdot 10 ^{-23} \ \si{\second} \)

\section{Mandelstam variables and scattering angle}
\textbf{a)} The Mandelstam varbiales are as follows:
\begin{align}
    s &= \left( p_1 + p_2 \right) ^2 = \left( p_3 + p_4 \right) ^2 \\
    t &= \left( p_1 - p_3 \right) ^2 = \left( p_2 - p_4 \right) ^2 \\
    u &= \left( p_1 - p_4 \right) ^2 = \left( p_2 - p_3 \right) ^2 
\end{align}
\textbf{b)} The square of a particle's four momentum is its mass (\(p_i^2 = m_i\)), and momentum
has to be conserved ( \( p_1 +p_2 = p_3 + p_4 \)). Therefore
\begin{align}
\nonumber s+t+u &= p_1^2+p_2^2+2p_1p_2 + p_1^2+p_3^2-2p_1p_3+p_1^2+p_4^2-2p_1p_4 \\
\nonumber&= m_1^2+m_2^2+m_3^2+m_4^2+2p_1^2+2p_1p_2-2p_1p_3-2p_1p_4 \\
\nonumber&= m_1^2+m_2^2+m_3^2+m_4^2+\underbrace{2p_1(p_1+p_2-p_3-p_4)}_{=0}\\
&= m_1^2+m_2^2+m_3^2+m_4^2
\end{align}

\end{document}
